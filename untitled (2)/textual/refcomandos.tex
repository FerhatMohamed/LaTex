% ~~~~~~~~~~~~~~~~~~~~~~~~~~~~~~~~~~~~~~~~~~~~~~~~~~~~~~~~~~~~~~~~~~~~~
%  File   : rt_cap1
% ~~~~~~~~~~~~~~~~~~~~~~~~~~~~~~~~~~~~~~~~~~~~~~~~~~~~~~~~~~~~~~~~~~~~~

Os seguintes comandos personalizados estão disponíveis para facilitar a digitação no \LaTeX:

\begin{itemize}
 \item \verb|\AI|: \AI\ -- imprime a expressão ``\AI''
 \item \verb|\ai|: \ai\ -- imprime a expressão ``\ai''
 \item \verb|\cepunb|: \cepunb\ -- imprime o CEP da \unb 
 \item \verb|\CI|: \CI\ -- imprime a expressão ``\CI''
 \item \verb|\ci|: \ci\ -- imprime a expressão ``\ci''
 \item \verb|\cnpjfub|: \cnpjfub\ -- imprime o CNPJ da \fub
 \item \verb|\CPAI|: \CPAI\ -- imprime a expressão ``\CPAI''
 \item \verb|\cpai|: \cpai\ -- imprime a expressão ``\cpai''
 \item \verb|\cpaiemail|: \cpaiemail\ -- imprime o e-mail do \cpai\ (com hiperlinks)
 \item \verb|\cpaiend|: \cpaiend\ -- imprime o endereço do \cpai 
 \item \verb|\cpaitelefone|: \cpaitelefone\ -- imprime o telefone do \cpai 
 \item \verb|\cpaiurl|: \cpaiurl\ -- imprime a URL do \cpai (com hiperlink)
 \item \verb|\diretorcpai|: \diretorcpai\ -- imprime o nome do Diretor do \cpai 
 \item \verb|\filiacaocpai|: \filiacaocpai\ -- imprime a filiação do \cpai 
 \item \verb|\fubend|: \fubend\ -- imprime o endereço da \fub 
 \item \verb|\fub|: \fub\ -- imprime a expressão ``\fub''
 \item \verb|\FUB|: \FUB\  -- imprime a expressão ``\FUB''
 \item \verb|\iAI|: \iAI\  -- imprime a expressão ``\iAI''
 \item \verb|\MAIA|: \MAIA\ -- imprime a expressão ``\MAIA''
 \item \verb|\maia|: \maia\  -- imprime a expressão ``\maia''
 \item \verb|\Pnrh|: \Pnrh\  -- imprime a expressão ``\Pnrh''
 \item \verb|\pnrh|: \pnrh\  -- imprime a expressão ``\pnrh''
 \item \verb|\PNRS|: \PNRS\ -- imprime a expressão ``\PNRS''
 \item \verb|\pnrs|: \pnrs\  -- imprime a expressão ``\pnrs''
 \item \verb|\reitor|: \reitor\ -- imprime o nome do Reitor da \unb 
 \item \verb|\rhid|: \rhid\ -- imprime a expressão ``\rhid''
 \item \verb|\rtautor|: \rtautor\ -- imprime o autor do relatório (altere no arquivo \verb|sty/metadados_rt.sty|)
 \item \verb|\rtdata|: \rtdata\ --   imrpime a data da publicação do relatório (Mês e Ano) (altere no arquivo \verb|sty/metadados_rt.sty|)
 \item \verb|\rtkeyworda|: \rtkeyworda\ -- primeira palavra-chave do relatório (altere no arquivo \verb|sty/metadados_rt.sty|)
 \item \verb|\rtkeywordb|: \rtkeywordb\ -- segunda palavra-chave do relatório (altere no arquivo \verb|sty/metadados_rt.sty|)
 \item \verb|\rtkeywordc|: \rtkeywordc\ -- terceira palavra-chave do relatório (altere no arquivo \verb|sty/metadados_rt.sty|)
 \item \verb|rtlider|: \rtlider\ -- imprime o nome do pesquisador líder do projeto (altere no arquivo \verb|sty/metadados_rt.sty|)
 \item \verb|\rtlocal|: \rtlocal\ -- imprime o local da publicação do Relatório (altere no arquivo \verb|sty/metadados_rt.sty|)
 \item \verb|\rtprojeto|: \rtprojeto\ -- imprime o projeto ao qual está relacionado o Relatório (altere no arquivo \verb|sty/metadados_rt.sty|)
 \item \verb|\rttitulo|: \rttitulo\ --   imprime o título do relatório (altere no arquivo \verb|sty/metadados_rt.sty|)
 \item \verb|\rtversao|: \rtversao\ --   imprime o número da versão do relatório (altere no arquivo \verb|sty/metadados_rt.sty|)
 \item \verb|\siglaai|: \siglaai\ -- imprime a expressão ``\siglaai''
 \item \verb|\siglacpai|: \siglacpai\ -- imprime a expressão ``\siglacpai''
 \item \verb|\siglafub|: \siglafub\  -- imprime a expressão ``\siglafub''
 \item \verb|\siglati|: \siglati\  -- imprime a expressão ``\siglati''
 \item \verb|\sinf|: \sinf\  -- imprime a expressão ``\sinf''
 \item \verb|\Sinf|: \Sinf\  -- imprime a expressão ``\Sinf''
 \item \verb|\SINIR|: \SINIR\ -- imprime a expressão ``\SINIR''
 \item \verb|\sinir|: \sinir\  -- imprime a expressão ``\sinir''
 \item \verb|\snirh|: \snirh\ -- imprime a expressão ``\snirh''
 \item \verb|\TAI|: \TAI\ -- imprime a expressão ``\TAI''
 \item \verb|\tai|: \tai\ -- imprime a expressão ``\tai''
 \item \verb|\TI|: \TI\ -- imprime a expressão ``\TI''
 \item \verb|\ti|: \ti\  -- imprime a expressão ``\ti''
 \item \verb|\tr|: \tr\ -- imprime a expressão ``\tr''
 \item \verb|\UnB|: \UnB\  -- imprime a expressão ``\UnB''
 \item \verb|\unb|: \unb\  -- imprime a expressão ``\unb''
 \item \verb|\vicereitor|: \vicereitor\ -- imprime o nome do Vice-Reitor
\end{itemize}