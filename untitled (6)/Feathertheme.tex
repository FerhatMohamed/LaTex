\documentclass[10pt]{beamer}
\usetheme[
%%% option passed to the outer theme
%    progressstyle=fixedCircCnt,   % fixedCircCnt, movingCircCnt (moving is deault)
  ]{Feather}
  
% If you want to change the colors of the various elements in the theme, edit and uncomment the following lines

% Change the bar colors:
%\setbeamercolor{Feather}{fg=red!20,bg=red}

% Change the color of the structural elements:
%\setbeamercolor{structure}{fg=red}

% Change the frame title text color:
%\setbeamercolor{frametitle}{fg=blue}

% Change the normal text color background:
%\setbeamercolor{normal text}{fg=black,bg=gray!10}

%-------------------------------------------------------
% INCLUDE PACKAGES
%-------------------------------------------------------

\usepackage[utf8]{inputenc}
\usepackage[english]{babel}
\usepackage[T1]{fontenc}
\usepackage{helvet}

%-------------------------------------------------------
% DEFFINING AND REDEFINING COMMANDS
%-------------------------------------------------------

% colored hyperlinks
\newcommand{\chref}[2]{
  \href{#1}{{\usebeamercolor[bg]{Feather}#2}}
}

%-------------------------------------------------------
% INFORMATION IN THE TITLE PAGE
%-------------------------------------------------------

\title[] % [] is optional - is placed on the bottom of the sidebar on every slide
{ % is placed on the title page
      \textbf{Trabajo monográfico utilizando herramientas colaborativas: Google}
}


\author[Oscar y Enrique]
{      
Enrique González Cantón \\
Óscar David Gómez López \\
}

\institute[]
{
      Universidad de Granada\\
  
  %there must be an empty line above this line - otherwise some unwanted space is added between the university and the country (I do not know why;( )
}

\date{\today}

%-------------------------------------------------------
% THE BODY OF THE PRESENTATION
%-------------------------------------------------------

\begin{document}

%-------------------------------------------------------
% THE TITLEPAGE
%-------------------------------------------------------

{\1% % this is the name of the PDF file for the background
\begin{frame}[plain,noframenumbering] % the plain option removes the header from the title page, noframenumbering removes the numbering of this frame only
  \titlepage % call the title page information from above
\end{frame}}


%\begin{frame}{Content}{}
%\tableofcontents
%\end{frame}

%-------------------------------------------------------
\section{Actividad}
%-------------------------------------------------------
\begin{frame}{Actividad}{}
%-------------------------------------------------------

Trabajo monográfico sobre cualquiera de los temas de la asignatura utilizando las herramientas de trabajo colaborativo de Google. \\
    \ \\
    Grupos de 3-4 personas \\
    \ \\
     \textbf{Evaluación:} \\
     \ \\
     1. Calidad del trabajo monográfico (60\%)\\
	 2. Uso de las herramientas de trabajo colaborativo de Google (40\%)
\end{frame}

\begin{frame}{Pasos a seguir}{}
1. Crear de una cuenta de google (si no se tiene ya).\\ \ \\
2. Compartir tu correo con tus compañeros.\\ \ \\
3. Crear tu primer documento.\\ \ \\
4. ¡Empieza a trabajar!\\
\end{frame}

\begin{frame}{Ejemplos de Herramientas Complementarias}{}

1. Google Calendar: para planificar tus tareas\\ \ \\
2. Google Presentaciones: elabora una bonita presentación de tu trabajo.\\ \ \\
3. Google Drive: para compartir el material con tus compañeros\\ \ \\
4. Google Hangouts: para chatear y debatir cuando no podaís reuniros.\\ \ \\
5. Google Scholar: para profundizar tus conocimientos de la asignatura\\ \ \\

\end{frame}
\end{document}